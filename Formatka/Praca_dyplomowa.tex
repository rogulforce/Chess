% !TeX spellcheck = pl_PL-Polish
%%%%%%%%%%%%%%%%%%%%%%%%%%%%%%%%%%%%%%%%%%%%%%%%%%%%%%%%%
% Niniejszy plik przedstawia przykładowy skład 
% pracy dyplomowej na Wydziale Matematyki PWr. 
% 
% Autorzy: 
% Damian Fafuła
% Michał Kijaczko
% Jakub Michalczak
% Maciej Miśta
% Dagmara Nowak
% Tomasz Skalski
% Wojciech Słomian
%
%% Data utworzenia: 8.05.2018
% Numer wersji: 1
%
% Poniższą formatkę można rozpowszechniać i edytować 
% pod warunkiem zachowania numeru wersji, 
% informacji o autorach i dodaniu informacji 
% o wprowadzonych zmianach.
%
%%%%%%%%%%%%%%%%%%%%%%%%%%%%%%%%%%%%%%%%%%%%%%%%%%%%%%%%%
% Domyślną opcją jest: praca magisterska, język polski.
% W przypadku pracy pisanej w języku angielskim dodajemy 
% opcję [english].
% Dla pracy licencjackiej dodajemy opcję [licencjacka].
% Dla pracy inżynierskiej dodajemy opcję [inzynierska].
% Dopuszczalne są podwójne opcje, np. [licencjacka, english].
% Opcje dodajemy w kwadratowym nawiasie przy \documentclass.
%
%
%%%%%%%%%%%%%%%%%%%%%%%%%%%%%%%%%%%%%%%%%%%%%%%%%%%%%%%%%
\documentclass[inzynierska]{pwr_wmat_praca_dyplomowa}
%%%%%%%%%%%%%%%%%%%%%%%%%%%%%%%%%%%%%%%%%%%%%%%%%%%%%%%%%
%              DANE DO PRACY
%
% W przypadku pracy dyplomowej w języku angielskim nie jest konieczne 
% wypełnianie pól: \tytul{}, \kierunek{}, \specjalnosc{}, 
%                  \streszczenie{}, \slowakluczowe{}.
%%%%%%%%%%%%%%%%%%%%%%%%%%%%%%%%%%%%%%%%%%%%%%%%%%%%%%%%%
%
% Imię i nazwisko autora
\autor{Piotr Rogula}
%
% Tytuł pracy dyplomowej 
\tytul{Analiza statystyczna czasów na wykonywanie ruchów w
	szachach} 
\tytulang{Statistical analysis of times for making moves in chess}
%
% Tytuł / stopień / imię i nazwisko opiekuna
\opiekun{Prof. dr hab. inż. Marcin Magdziarz}
%
% Kierunek studiów wybieramy spośród następujących:
% 1) Matematyka
% 2) Matematyka i Statystyka
% 3) Matematyka stosowana
\kierunekstudiow{Matematyka Stosowana}
%
% Kierunek studiów po angielsku wybieramy spośród następujących:
% 1) Mathematics
% 2) Mathematics and Statistics
% 3) Applied Mathematics
\kierunekstudiowang{Applied Mathematics}
%
% Specjalność wybieramy spośród następujących: 
% KIERUNEK: Matematyka
% 1) Matematyka teoretyczna,
% 2) Statystyka matematyczna,
% 3) Matematyka finansowa i ubezpieczeniowa,
%
% KIERUNEK: Matematyka i Statystyka
% 4) Matematyka,
% 5) Statystyka i analiza danych, 
%
% 6) -- (w przypadku braku specjalizacji).
\specjalnosc{--} 
%
% Specjalność w języku angielskim wybieramy spośród następujących:
% KIERUNEK: Matematyka
% 1) Theoretical Mathematics,
% 2) Mathematical Statistics,
% 3) Financial and Actuarial Mathematics,
%
% KIERUNEK: Matematyka i Statystyka
% 4) Mathematics,
% 5) Statistics and Data Analysis,
%
% KIERUNEK: Applied Mathematics
% 6) Financial and Actuarial Mathematics, 
% 7) Mathematics for Industry and Commerce,
% 8) Computational Mathematics,
% 9) Modelling, Simulation and Optimization.
%
% 10) -- (w przypadku braku specjalizacji).
\specjalnoscang{--} 
%
% Krótkie streszczenia po polsku i angielsku
% - nie dłuższe niż 530 znaków.
\streszczenie{ROBOCZE: W pracy przeanalizowana zostanie zależność między czasami wykonania poszczególnych ruchów w szachach, a ich zgodnością z silnikiem szachowym. Zbadany zostanie rozkład tych czasów w zależności od poziomu graczy, na tej podstawie obliczone zostanie prawdopodobieństwo wykonania błędnego ruchu (max 530 znaków).}
\streszczenieang{Tutaj piszemy krótkie streszczenie pracy w języku angielskim (max 530 znaków).}
%
% Podajemy najważniejsze słowa kluczowe po polsku i angielsku
% - w obu przypadkach, nie więcej niż 150 znaków.
\slowakluczowe{ROBOCZE: analiza statystyczna, szachy, korelacja (max 150 znaków).}  
\slowakluczoweang{tutaj podajemy najważniejsze słowa kluczowe w języku angielskim (łącznie nie powinny być dłuższe niż 150 znaków)}
%
%
%%%%%%%%%%%%%%%%%%%%%%%%%%%%%%%%%%%%%%%%%%%%%%%%%%%%%%%%%
% Definicje, lematy, twierdzenia, przykłady i wnioski
% Komendy wywołujące twierdzenia, definicje, itd., 
% czyli 'theorem', 'definition', 'corollary', itd., 
% można zmienić wedle uznania.
\theoremstyle{plain}
\newtheorem{theorem}{Twierdzenie}
\numberwithin{theorem}{chapter}
\newtheorem{lemma}[theorem]{Lemat} 
\newtheorem{corollary}[theorem]{Wniosek}
\newtheorem{fact}[theorem]{Fakt}
\theoremstyle{definition}
\numberwithin{theorem}{chapter}
\newtheorem{definition}[theorem]{Definicja} 
\newtheorem{example}[theorem]{Przykład}
\newtheorem{note}[theorem]{Uwaga}
%%%%%%%%%%%%%%%%%%%%%%%%%%%%%%%%%%%%%%%%%%%%%%%%%%%%%%%%%


%%%%%%%%%%%%%%%%%%%%%%%%%%%%%%%%%%%%%%%%%%%%%%%%%%%%%%%%%
%%%%%%%%%%%%%%%%%%%%%%%%%%%%%%%%%%%%%%%%%%%%%%%%%%%%%%%%%
\begin{document}
\bibliographystyle{plain}
\frontmatter
\maketitle
\mainmatter
\tableofcontents
%\listoffigures
%\listoftables

{\backmatter \chapter{Wstęp}}
\textbf{We wstępie zapowiadamy, o czym będzie praca. Próbujemy zachęcić czytelnika do dalszej lektury, np. krótko informując, dlaczego wybraliśmy właśnie ten temat i co nas w nim zainteresowało.}

Wraz z rozwojem technologii komputerowej, rozpoczęła się nowa era szachów. Technologia korzystając z dużej mocy obliczeniowej, bezpowrotnie wyprzedziła człowieka w grach deterministycznych,\textbf{ a ostatnio też i tych niedeterministycznych (?)}. Profesjonalni szachiści zaczęli wykorzystywać nowe strategie korzystając z coraz lepszych silników szachowych. Silniki te oceniają wprowadzoną pozycję pod kątem przewagi jednej ze stron. 

W dobie internetu gra w szachy stała się dużo wygodniejsza niż przed laty. Ludzie grają w różnych miejscach i praktycznie o każdej porze. W związku z tym dużo większą popularnością zaczęły cieszyć się szachy szybkie, czyli takie, w których każdy z zawodników ma relatywnie mało czasu na wykonanie wszystkich ruchów. Wiąże się to z dużo większym znaczeniem dysponowania czasem w trakcie gry. W każdym ruchu zawodnik musi ustalić równowagę pomiędzy dokładnością ruchu, a czasem, który jest w stanie na ten ruch poświęcić. 

Przedmiotem badań tej pracy jest analiza zależności między dokładnością ruchu, a czasem, który został na niego poświęcony dla zawodników prezentujących różny poziom umiejętności i dla różnych formatów czasowych.\textbf{ Zbadanie takiej zależności może pozwolić na określenie optymalnego czasu na wykonanie ruchu dla odpowiedniej fazy gry i formatu czasowego.}

\textbf{DODAĆ TUTAJ TROCHE I OGÓLNY CEL}\\


\textbf{W PIERWSZEJ CZĘŚCI - ZAGADNIENIA TEORETYCZNE DOTYCZĄCE SZACHÓW}\\
W pierwszej części pracy przedstawione i wyjaśnione zostaną podstawowe zagadnienia teoretyczne związane z szachami, systemami rankingowymi i silnikami szachowymi. 


\textbf{W DRUGIEJ CZĘŚCI ZAGADNIENIA TEORETYCZNE ZE STATYSTYKI I METODOLOGII}\\
Kolejna część pracy opowiada o zagadnieniach teoretycznych z dziedziny statystyki, zastosowanych w analizie przedstawionych problemów.
 
 
\textbf{PÓŹNIEJ DOKŁADNE SFORMUOWANIE PROBLEMU}\\
Główna część pracy zawiera przedstawienie...

\textbf{DOKŁADNE ROZWIĄZANIE PROBLEMU}\\
a póżniej rozwiązanie...

\textbf{PODSUMOWANIE}
Ostatnia część pracy podsumowanie, wnioski...

\chapter{ZAGADNIENIE TEORETYCZNE I - DOTYCZĄCE SZACHÓW}
Niniejszy rozdział poświęcony zostanie zagadnieniom teoretycznym dotyczącym szachów, używanych systemów rankingowych oraz działaniu silników szachowych.
\section{OPISAĆ ZASADY GRY W SZACHY ??}
Początki szachów nie są znane, jednak ich historia trwa już ok. 1500 lat i zaczyna się w Indiach. Na przestrzeni wieków zasady gry w szachy były wielokrotnie zmieniane. Powszechnie stosowane przepisy pochodzą z roku 1851.\\

\textbf{krótko na czym polegają szachy i cite gdzie można znaleźć pełne przepisy,isbn:002028540X}\\

Gra odbywa się na kwadratowej planszy o wymiarach 8 na 8 pól. Każdy gracz posiada 16 figur, ustawionych w pozycji startowej i stojących po przeciwnych stronach szachownicy. Zawodnicy wykonują na przemian ruchy dowolną ze swoich figur, zgodnie z jej zasadami poruszania się. W trakcie tury zawodnikowi upływa czas ustalony przed grą. W niektórych wariantach gracz otrzymuje też niewielką ilość czasu za wykonanie każdego ruchu.\\

Wygrana następuje, gdy król jednego z graczy jest atakowany i nie można w legalny sposób nim ruszyć, ani zasłonić jedną ze swoich figur przed atakiem. Pozycję taką nazywa się ,,matem'' Drugim sposobem na wygranie jest skończenie się czasu jednego z graczy, niezależnie od sytuacji na szachownicy. 
Gra może też zakończyć się remisem. Następuje on, gdy żaden z graczy nie ma na planszy figur, które mogą pozwolić na ,,zamatowanie'' przeciwnika. Inną możliwością jest trzykrotne powtórzenie się na planszy tej samej pozycji. Do remisu doprowadza też sytuacja, w której jednemu z graczy zakończył się czas, a jego przeciwnik nie ma figur pozwalających na wygraną lub gdy jeden z graczy nie ma możliwości wykonania żadnego legalnego ruchu, a jego król nie jest atakowany.

W związku z możliwością przegranej poprzez upłynięcie czasu na zegarze, zawodnicy muszą indywidualnie określić podczas gry, ile czasu są w stanie poświęcić danemu ruchowi, tak by nie stracić na niego zbyt wiele czasu, ale też, żeby ruch był jak najlepszy.

\section{OPISAĆ NOTACJĘ szachową????? - nie będę w sumie nic z nią robić, ale jest}

\section{OPISAĆ szachowy system Glicko-2 (oparty na rozkładzie normalnym)}
\textbf{opisać ogólnie troche historii o systemach rankingowych?}
System rankingowy ELO został zaprezentowany w latach pięćdziesiątych dwudziestego wieku przez Węgierskiego fizyka i szachistę Arpada Elo (1903-1992) \textbf{[CITE]}. Początkowo był używany jedynie w szachach, jednak wraz ze wzrostem jego popularności zaczął być stosowany również w innych \textbf{rozgrywkach}. System ten jest pierwszym systemem mającym podłoże probabilistyczne i jest oparty na rozkładzie normalnym z ustaloną średnią. Przyznaje odpowiednią liczbę punktów zwycięzcy rozgrywki i odbiera przegranemu bazując na różnicy między ich aktualnym rankingiem.


System Glicko-2 używany przez stronę \textbf{Lichess.com}, na danych której oparta jest niniejsza praca, opracowany został przez Marka Glickmana jako ulepszenie systemu ELO. Podstawową zmianą jest uwzględnienie historycznych wyników każdego z zawodników w celu ustalenia wariancji aktualnego rankingu. Glickman w swojej pracy z roku 1998 \textbf{[cite]} przedstawia problem dwóch graczy o takim samym rankingu, z których jeden gra regularnie, a drugi wrócił do gry po długiej przerwie. System Glicko-2 przyznając punkt za grę bierze pod uwagę wiarygodność każdego z rankingów. Zawodnikowi grającemu regularnie zostanie przyznane bądź odebrane mniej punktów ze względu na duże potencjalne odchylenie rankingu przeciwnika od zadeklarowanej wartości. Innymi słowy, w miarę zwiększania się liczby partii gracza, przedział ufności dla jego realnego rankingu zawęża się i przypisany mu ranking zbiega do realnego poziomu i wiarygoność przypisanego rankingu jest uwzględnian w przyznawaniu i odbieraniu punktów zawodnikom po zakończeniu partii.

\textbf{WRZUCIĆ MATEMATYKĘ STOJĄCĄ ZA GLICKO-2???}
x
\subsection{z uwzględnieniem ELO na platformie Lichess, z której bierzemy dane}

\textbf{Tutaj jakiś wykresik może jak wygląda rozkład rankingu zawodników na platformie Lichess ???}
\section{Funkcja oceny}
Przed przystąpieniem do opisania funkcji, należy wytłumaczyć działanie silnika szachowego, który dokonuje oceny pozycji.
\subsection{Stockfish}
Stockfish jest jednym z najlepszych i najpopularniejszym obecnie używanym silnikiem szachowym, zaprojektowanym przez Marco Costalba, Joona Kiiski, Gary Linscott, Tord Romstad, Stéphane Nicolet, Stefan Geschwentner, and Joost VandeVondele i stale ulepszany jako oprogramowanie typu open-source. Strona \textbf{Lichess.com}\cite{stockfish_lichess} wykorzystuje go do analizy i oceny aktualnej pozycji

Stockfish poprzez przeszukiwanie wg strategii mini-max z odcięciem, za pomocą algorytmu alfa-beta, analizuje legalne (czyli następujące po ruchu zgodnym z zasadami gry) pozycje, które mogą wyniknąć z aktualnej sytuacji na szachownicy. Dobierają na podstawie najlepszego możliwego zestawu ruchów (zakłada się, że każdy z graczy wykona najlepszy w ocenie silnika ruch) pozycje, które wystąpią dla określonej głębokości (głębokość 18 oznacza 18 ruchów białych i 18 czarnych wykonanych zaczynając z analizowanej pozycji) i na ich podstawie ocenia aktualną pozycję, określając przewagę jednego z graczy

\subsection{Ewaluacja}
Wspomniana wcześniej ewaluacja, wyliczana przez silnik szachowy jest wynikiem liniowej funkcji 
ważonej sumy cech, na którą składają się między innymi:\\
$f_b,f_c$ oznaczających wartość figur odpowiednio białych i czarnych\\
$k_b,k_c$ oznaczających bezpieczeństwo króla odpowiednio białych i czarnych\\
$m_b,m_c$ oznaczających mobilność figur odpowiednio białych i czarnych\\
$z_b,z_c$ oznaczających potencjalne zagrożenia wykonane odpowiednio białych i czarnych\\


Funkcję można dla zapewnienia intuicji zapisać w uproszeniu:
\begin{equation}
	f(f_b,f_c,k_b,k_c,m_b,m_c,\dots)=c_1(f_b-f_c)+c_2(k_b-k_c)+c_3(m_b-m_c)+\dots
\end{equation}
gdzie:
$c_i$ są stałymi określającymi wagę danej pary zmiennych.

Wraz ze wzrostem wartości funkcji zwiększa się przewaga białych, natomiast wraz z jej spadkiem, przewaga czarnych. Wartość wynosząca 0 oznacza stan równowagi. Dodatkowo, w przypadku nieuniknionego zwycięstwa jednej ze stron w \textit{n} ruchach, wynikiem funkcji zamiast odpowiedniej wartości liczbowej jest tekst \textit{\#-n} w przypadku wygranej czarnych lub \textit{\#n} w przypadku wygranej białych, oznaczający nieuchronną wygraną jednego z graczy po wykonaniu \textit{n} odpowiednich ruchów.


\subsubsection{rodzaje błędów szachowych}
\textbf{OPISAĆ DEFINICJE INNACURACY, MISTAKE I BLUNDER}

% https://en.wikipedia.org/wiki/Chess_annotation_symbols

W notacji szachowej obok zapisanego ruchu mogą pojawić się symbole określające jakość danego ruchu. Dla analizowanych danych, ruch oceniany jest przez silnik szachowy za pomocą skomplikowanych algorytmów.


(opisać te algorytmy w urposzczeniu - tj. blunder gdy delta eval jest wieksze niż pewna wartość (np. 2), ale tylko gdy sytuajca nie jest przesądzona,
np zmiana 0 -> 2.5 BLUNDER, zmiana 22 -> 25, nie BLUNDER)

?? - duży błąd
 
OPIS
 
 
 ? - pomyłka
 
 OPIS
 
 ?! - WĄTPLIWE POSUNIĘCIE
 
 !? - posunięcie zasługujące na uwagę
 
 ! -  bardzo dobre posunięcie
 
 !! wyśmienite posunięcie

LEPIEJ OPISANE NA WIKI ANG 




\chapter{ZAGADNIENIE TEORETYCZNE II - użyte metody, teoria stojąca za rozwiązaniami problemów}
Niniejszy rozdział poświęcony zostanie zagadnieniom teoretycznym z dziedziny statystyki, zastosowanych w analizie przedstawionych problemów.
\section{zagadnienie 1...}

\chapter{sformuowanie problemów analitycznych, które chce zbadać}

\chapter{analiza / rozwiązanie problemów}
\section{Dane}
Dane potrzebne do analizy zostały pobrane z platformy Lichess \cite{lichess}. Są one przechowywane w plikach o rozmiarze kilkudziesięciu Gb. Każdy z nich zawiera wszystkie gry rozegrane na platformie w maju 2019 roku. Ponadto, ok. 7\% gier zostało wcześniej przeanalizowane przez silnik szachowy Stockfish i posiadają dane punktowe o nazwie \textit{Eval}, określające unormowaną przewagę jednego z graczy oraz określenie części ruchów przez silnik jako błędne. Dodatkowo silnik określa jakość wykonanego ruchu wg własnych kryteriów. Przykładowy zapis jednej takiej gry został zaprezentowany na rysunku \ref{rys:zapis_gry}. Informacje potrzebne do analizy to:
\begin{itemize}
	\item WhiteElo -- ranking białych
	\item BlackElo -- ranking czarnych
	\item TimeControl -- czas na wykonanie ruchów każdego z graczy w formacie  ,,sekundy + sekundy dodane za wykonanie ruchu''
	\item \% eval -- aktualna przewaga jednej ze stron
	\item zapis partii ze wskazaniem błędnych ruchów w ocenie silnika
	\item \% clk -- pozostały czas w formacie ,,godziny : minuty : sekundy''
	\item result -- wynik partii
\end{itemize}

\begin{figure}[H]
	\centering
	\includegraphics[width=\textwidth]{zapis_gry.png}
	\caption{Przykładowy zapis jednej partii, zmienić na aktualny}
	\label{rys:zapis_gry} 
\end{figure}



\subsection{Odfiltrowanie danych}

Pierwszym krokiem potrzebnym do wykonania analizy jest odfiltrowanie danych.
Po podzieleniu tekstu na kolejne gry, odfiltrowane zostały jedynie te, które zostały wcześniej ocenione przez silnik. Następnie dla każdej z nich przeanalizowany został każdy ruch, do którego zostały przypisane następujące atrybuty:
\begin{itemize}
	\item score -- ocena ruchu wg silnika Stockfish
	\item WhiteElo -- ranking białych
	\item BlackElo -- ranking czarnych
	\item TimeControl -- czas na wykonanie ruchów każdego z graczy w formacie  ,,sekundy + sekundy dodane za wykonanie ruchu''
	\item color -- gracz, wykonujący dany ruch
	\item move -- numer ruchu w danej partii
	\item result - wynik partii
\end{itemize}

Stworzona baza zawiera 9 737 663 posunięć ze 154 981 gier, co daje średnią 31.41 ruchu na grę (jeden ruch oznacza posunięcie białych i czarnych). Fragment bazy przedstawiony został na rysunku \ref{rys:baza_ruchow}
\begin{figure}[H]
	\centering
	\includegraphics[width=\textwidth]{danee.png}
	\caption{zmienić żeby nie było whiteratingdif}
	\label{rys:baza_ruchow}
\end{figure}



\section{Analiza pierwszego problemu}
Pierwszym problemem, który zostanie poruszony jest zbadanie statystycznej zależności jakości wykonanego ruchu wg oceny silnika Stockfish od czasu potrzebnego na jego wykonanie. W celu uzyskania bardziej precyzyjnych wyników pierwsze 4 ruchy (zarówno białych jak i czarnych) nie będą brane pod uwagę. Są one elementem teorii otwarć szachowych \textbf{ref ISBN 0-19-280049-3}, w związku z czym wykonywane są zazwyczaj bardzo szybko i ze znikomą szansą popełnienia błędu, co może spowodować zaburzenie danych.

\begin{table}[H]
	\caption{baza ruchów z gier na portalu Lichess z maja 2019, 6 najpopularniejszych formatów}
	\centering
	\begin{tabular}{|l|r|}
		\hline
		\multirow{2}{*}{\begin{tabular}[c]{@{}l@{}}Format czasowy \\ sekundy+sekundy przyznawane po wykonaniu ruchu\end{tabular}} & \multirow{2}{*}{\begin{tabular}[c]{@{}l@{}}liczba ruchów \\ w bazie\end{tabular}} \\
		&                                                                                   \\ \hline
		600+0 & 1 813 841\\
		300+0 & 1 409 904\\
		\hphantom{0}60+0 & 1 251 388 \\
		900+15 & 1 104 458 \\
		180+0 & 1 057 292\\
		300+3 & 941 262\\  \hline
	\end{tabular}
	\label{tab:formaty} 
\end{table}

Analizowane będzie sześć najpopularniejszych formatów granych na platformie Lichess. Są one przedstawione w tabeli \ref{tab:formaty}. Przy analizie z uwzględnieniem rankingu graczy, zastosowane będą podziały kwartylowe wg rankingu białych, rozkład rankingu wraz z zaznaczonymi kwartylami pokazany jest na rysunku \ref{rys:rozklad_elo}. Jak widać zbiega on do rozkładu normalnego. Anomalia dla rankingu 1500 związana jest z tym, że jest to ranking przyznawany każdemu graczowi przy jego pierwszej rozgrywce na platformie.

\begin{figure}[H]
	\centering
	%\includegraphics[width=\textwidth]{sample60.png}
	\caption{tutaj grafika rozkładu ELO}
	\label{rys:rozklad_elo}
\end{figure}





TUTAJ rozkłady, 

np dla formatu 60+0 (60 sekund, brak dodawanego czasu po wykonaniu ruchu)

\begin{figure}[H]
	\centering
	\includegraphics[width=\textwidth]{sample60.png}
	\caption{xxx}\label{xxx}
\end{figure}


oś x -> czas

oś y -> nieznormalizowana liczba ruchów typu 'blunder' (te najcięższe pomyłki)\newline


rozkład gamma... (?)\newline 




TO DO: 

sprawdzenie zmian dla rankingu graczy, różnicy rankingu graczy

porównanie z czasem na wykonanie każdego ruchu\newline



TO DO: 

czy różnica pomiędzy formatem z dodawanym czasem po ruchu, a bez dodawanego czasu jest widoczna?



\section{analiza drugiego problemu...}
tutaj statystyczne prawdopodobieństwo wykonania złego ruchu pod warunkiem poświęceniu mu konkretnego czasu,\newline

np, w formacie czasowym 60+0 na ruch zostały poświęcone 4 sekundy, jaka jest szansa, że został popełniony błąd \newline
\begin{figure}[H]
	\centering
	\includegraphics[width=\textwidth]{p_od_czasu.png}
	\caption{aaa}\label{aaa}
\end{figure}
CEL: 
ile powinno się poświęcić czasu na ruch by obniżyć prawdopodobieństwo wykonania błędu?\newline

\textbf{Tego jeszcze nie analizowałem}




\section{analiza trzeciego problemu...  and so on...}
w którym ruchu jest największa szansa na popełnienie błędu?
dla wszystkich rang
\begin{figure}[H]
	\centering
	\includegraphics[width=\textwidth]{p_od_ruchu.png}
	\caption{aaa}\label{aaa}
\end{figure}
\chapter{wnioski, podsumowanie}

\chapter{tabelka}
Tabela \ref{tab:przykladowa} 
\begin{table}[H]
	\caption{Podstawowa Tabela}
	\centering
	\begin{tabular}{ccc}
		\hline
		\hline                       
		Państwo & PKB (w milionach USD )& Stopa bezrobocia  \\  [0.5ex] 
		\hline 
		Stany Zjednoczone & 75 278 049 & 4,60\%  \\
		Chiny & 11 218 281 & 4,10\%   \\
		Japonia & 4 938 644 & 3,10\%  \\
		Niemcy & 3 466 639 & 6,00\%   \\
		Wielka Brytania & 2 629 188 & 4,60\%  \\ [1ex]  
		\hline 
	\end{tabular}
	\caption*{\textit{Źródło: opracowanie własne}}
	\label{tab:przykladowa2} 
\end{table}
\chapter{rysunek}
Rysunki do pracy dyplomowej należy wstawiać w sposób podobny do wstawiania tabel, z~zasadniczą różnicą polegającą na tym, że podpis powinno umieszczać się centralnie pod rysunkiem, a nie powyżej niego. Numeracja i sposób cytowania pozostają bez zmian, przy czym tabele i rysunki nie mają numeracji wspólnej, np. po Tabeli \ref{tab:przykladowa2} występuje Rysunek \ref{rys1} (o ile jest to pierwszy rysunek rozdziału pierwszego), a nie Rysunek $1.3$.

\begin{figure}[ht]
	
	\centering
	
	\includegraphics[scale=0.27]{logo_w13.jpg}
	\caption{Podstawowy Rysunek}\label{rys1}
\end{figure}
\label{rys:przykladowy} 


\chapter{Definicje, lematy, twierdzenia, przykłady i wnioski}
Definicje, lematy, twierdzenia, przykłady i wnioski piszemy w pracy tak:
\begin{definition}[Martyngał]
	Tu piszemy treść definicji martyngału.
\end{definition}
\begin{lemma}[]% w nawiasie kwadratowym można napisać jego nazwę
	Tu piszemy treść lematu.
\end{lemma}
\chapter{cytowanie}
Do cytowania używamy komendy \texttt{cite}. W nawiasie klamrowym podajemy klucz, którego użyliśmy w pliku \emph{bibliografia.bib}. Przykład: \cite{einstein} lub \cite[chap. 2]{latexcompanion}.

%{\backmatter \chapter{Podsumowanie}}
%Podsumowanie w pracach matematycznych nie jest obligatoryjne. Warto jednak na zakończenie krótko napisać, co udało nam się %zrobić w pracy, a czasem także o tym, czego nie udało się zrobić.

{\backmatter \chapter{Dodatek}}
Dodatek w pracach matematycznych również nie jest wymagany. Można w nim przedstawić np. jakiś dłuższy dowód, który z pewnych przyczyn pominęliśmy we właściwej części pracy lub (np. w przypadku prac statystycznych) umieścić dane, które analizowaliśmy.

%%%%%%%%%%%%%%%%%%%%%%%%%%%%%%%%%%%%%%%%%%%%%%%%%%%%%%%%%
% BIBLIOGRAFIA
% W tworzeniu bibliografii najlepiej korzystać z BibTex'a, 
% który jest częścią systemu Tex. W naszym przypadku funkcję 
% przechowalni literatury, do której się odwołujemy, pełni 
% plik bibliografia.bib. Nie musimy ręcznie dodawać nowych 
% pozycji do bibliografii. Możemy wejść np. na stronę 
% https://mathscinet.ams.org/mathscinet/index.html, 
% znaleźć odpowiednią pozycję, wybrać ją, a następnie zmienić 
% 'Select alternative format' na BibTeX, skopiować uzyskany 
% tekst, wkleić do pliku bibliografia.bib i skompilować. 
% Gotowe informacje do pliku bibliografia.bib można znaleźć 
% także na https://arxiv.org - gdy znajdziemy interesującą nas 
% pracę, szukamy 'References & Citations' i klikamy 'NASA ADS', 
% a potem 'Bibtex entry for this abstract' 
% i postępujemy tak jak wcześniej.
%%%%%%%%%%%%%%%%%%%%%%%%%%%%%%%%%%%%%%%%%%%%%%%%%%%%%%%%%
\newpage
% w nawiasie klamrowym wpisujemy nazwę pliku z bibliografią w formacie .bib
\bibliography{bibliografia} 
\end{document}