% !TeX spellcheck = de_DE
%%%%%%%%%%%%%%%%%%%%%%%%%%%%%%%%%%%%%%%%%%%%%%%%%%%%%%%%%
% Niniejszy plik przedstawia przykładowy skład 
% pracy dyplomowej na Wydziale Matematyki PWr. 
% 
% Autorzy: 
% Damian Fafuła
% Michał Kijaczko
% Jakub Michalczak
% Maciej Miśta
% Dagmara Nowak
% Tomasz Skalski
% Wojciech Słomian
%
%% Data utworzenia: 8.05.2018
% Numer wersji: 1
%
% Poniższą formatkę można rozpowszechniać i edytować 
% pod warunkiem zachowania numeru wersji, 
% informacji o autorach i dodaniu informacji 
% o wprowadzonych zmianach.
%
%%%%%%%%%%%%%%%%%%%%%%%%%%%%%%%%%%%%%%%%%%%%%%%%%%%%%%%%%
% Domyślną opcją jest: praca magisterska, język polski.
% W przypadku pracy pisanej w języku angielskim dodajemy 
% opcję [english].
% Dla pracy licencjackiej dodajemy opcję [licencjacka].
% Dla pracy inżynierskiej dodajemy opcję [inzynierska].
% Dopuszczalne są podwójne opcje, np. [licencjacka, english].
% Opcje dodajemy w kwadratowym nawiasie przy \documentclass.
%
%
%%%%%%%%%%%%%%%%%%%%%%%%%%%%%%%%%%%%%%%%%%%%%%%%%%%%%%%%%
\documentclass[inzynierska]{pwr_wmat_praca_dyplomowa}
%%%%%%%%%%%%%%%%%%%%%%%%%%%%%%%%%%%%%%%%%%%%%%%%%%%%%%%%%
%              DANE DO PRACY
%
% W przypadku pracy dyplomowej w języku angielskim nie jest konieczne 
% wypełnianie pól: \tytul{}, \kierunek{}, \specjalnosc{}, 
%                  \streszczenie{}, \slowakluczowe{}.
%%%%%%%%%%%%%%%%%%%%%%%%%%%%%%%%%%%%%%%%%%%%%%%%%%%%%%%%%
%
% Imię i nazwisko autora
\autor{Piotr Rogula}
%
% Tytuł pracy dyplomowej 
\tytul{Analiza statystyczna czasów na wykonywanie ruchów w
	szachach} 
\tytulang{Tytuł pracy dyplomowej w języku angielskim}
%
% Tytuł / stopień / imię i nazwisko opiekuna
\opiekun{Prof. dr hab. inż. Marcin Magdziarz}
%
% Kierunek studiów wybieramy spośród następujących:
% 1) Matematyka
% 2) Matematyka i Statystyka
% 3) Matematyka stosowana
\kierunekstudiow{Matematyka Stosowana}
%
% Kierunek studiów po angielsku wybieramy spośród następujących:
% 1) Mathematics
% 2) Mathematics and Statistics
% 3) Applied Mathematics
\kierunekstudiowang{Applied Mathematics}
%
% Specjalność wybieramy spośród następujących: 
% KIERUNEK: Matematyka
% 1) Matematyka teoretyczna,
% 2) Statystyka matematyczna,
% 3) Matematyka finansowa i ubezpieczeniowa,
%
% KIERUNEK: Matematyka i Statystyka
% 4) Matematyka,
% 5) Statystyka i analiza danych, 
%
% 6) -- (w przypadku braku specjalizacji).
\specjalnosc{--} 
%
% Specjalność w języku angielskim wybieramy spośród następujących:
% KIERUNEK: Matematyka
% 1) Theoretical Mathematics,
% 2) Mathematical Statistics,
% 3) Financial and Actuarial Mathematics,
%
% KIERUNEK: Matematyka i Statystyka
% 4) Mathematics,
% 5) Statistics and Data Analysis,
%
% KIERUNEK: Applied Mathematics
% 6) Financial and Actuarial Mathematics, 
% 7) Mathematics for Industry and Commerce,
% 8) Computational Mathematics,
% 9) Modelling, Simulation and Optimization.
%
% 10) -- (w przypadku braku specjalizacji).
\specjalnoscang{--} 
%
% Krótkie streszczenia po polsku i angielsku
% - nie dłuższe niż 530 znaków.
\streszczenie{Tutaj piszemy krótkie streszczenie pracy (nie powinno być dłuższe niż 530 znaków).}
\streszczenieang{Tutaj piszemy krótkie streszczenie pracy w języku angielskim (nie powinno być dłuższe niż 530 znaków).}
%
% Podajemy najważniejsze słowa kluczowe po polsku i angielsku
% - w obu przypadkach, nie więcej niż 150 znaków.
\slowakluczowe{tutaj podajemy najważniejsze słowa kluczowe (łącznie nie powinny być dłuższe niż 150 znaków).}  
\slowakluczoweang{tutaj podajemy najważniejsze słowa kluczowe w języku angielskim (łącznie nie powinny być dłuższe niż 150 znaków)}
%
%
%%%%%%%%%%%%%%%%%%%%%%%%%%%%%%%%%%%%%%%%%%%%%%%%%%%%%%%%%
% Definicje, lematy, twierdzenia, przykłady i wnioski
% Komendy wywołujące twierdzenia, definicje, itd., 
% czyli 'theorem', 'definition', 'corollary', itd., 
% można zmienić wedle uznania.
\theoremstyle{plain}
\newtheorem{theorem}{Twierdzenie}
\numberwithin{theorem}{chapter}
\newtheorem{lemma}[theorem]{Lemat} 
\newtheorem{corollary}[theorem]{Wniosek}
\newtheorem{fact}[theorem]{Fakt}
\theoremstyle{definition}
\numberwithin{theorem}{chapter}
\newtheorem{definition}[theorem]{Definicja} 
\newtheorem{example}[theorem]{Przykład}
\newtheorem{note}[theorem]{Uwaga}
%%%%%%%%%%%%%%%%%%%%%%%%%%%%%%%%%%%%%%%%%%%%%%%%%%%%%%%%%


%%%%%%%%%%%%%%%%%%%%%%%%%%%%%%%%%%%%%%%%%%%%%%%%%%%%%%%%%
%%%%%%%%%%%%%%%%%%%%%%%%%%%%%%%%%%%%%%%%%%%%%%%%%%%%%%%%%
\begin{document}
\bibliographystyle{plain}
\frontmatter
\maketitle
\mainmatter
\tableofcontents
%\listoffigures
%\listoftables

{\backmatter \chapter{Wstęp}}
We wstępie zapowiadamy, o czym będzie praca. Próbujemy zachęcić czytelnika do dalszej lektury, np. krótko informując, dlaczego wybraliśmy właśnie ten temat i co nas w nim zainteresowało.

\chapter{ZAGADNIENIE TEORETYCZNE I - DOTYCZĄCE SZACHÓW}
\section{OPISAĆ ZASADY GRY W SZACHY}
\section{OPISAĆ NOTACJĘ szachową}
\section{OPISAĆ szachowy system Glicko-2 (oparty na rozkładzie normalnym)}
\textbf{opisać ogólnie troche historii o systemach rankingowych?}
System rankingowy ELO został zaprezentowany w latach 50 XX wieku przez Węgierskiego fizyka i szachistę Arpada Elo (1903-1992) \textbf{[CITE]}. Początkowo był używany jedynie w szachach, jednak wraz ze wzrostem jego popularności zaczął być stosowany również w innych \textbf{rozgrywkach}. System ten jest pierwszym systemem mającym podłoże probabilistyczne i jest oparty na rozkładzie normalnym z ustaloną średnią. Przyznaje odpowiednią liczbę punktów zwycięzcy rozgrywki i odbiera przegranemu bazując na różnicy między ich aktualnym rankingiem.


System Glicko-2 używany przez stronę \textbf{Lichess.com}, na danych której oparta jest niniejsza praca, opracowany został przez Marka Glickmana jako ulepszenie systemu ELO. Podstawową zmianą jest uwzględnienie historycznych wyników każdego z zawodników w celu ustalenia wariancji aktualnego rankingu. Glickman w swojej pracy z roku 1998 \textbf{[cite]} przedstawia problem dwóch graczy o takim samym rankingu, z których jeden gra regularnie, a drugi wrócił po długiej przerwie. System Glicko-2 przyznając punkt za grę bierze pod uwagę wiarygodność każdego z rankingów. Zawodnikowi grającemu regularnie zostanie przyznane bądź odebrane mniej punktów ze względu na duże potencjalne odchylenie rankingu przeciwnika od zadeklarowanej wartości. Innymi słowy, w miarę zwiększania się liczby partii gracza, przedział ufności dla jego realnego rankingu zawęża się i przypisany mu ranking zbiega do realnego poziomu i ta wiarygoność przypisanego rankingu jest uwzględniana w zmianie punktów zawodników po zakończeniu partii.

\textbf{WRZUCIĆ MATEMATYKĘ STOJĄCĄ ZA GLICKO-2???}
\subsection{z uwzględnieniem ELO na platformie Lichess, z której bierzemy dane}
\section{Funkcja ewaluacji}
Przed przystąpieniem do opisania funkcji, należy wytłumaczyć działanie silnika szachowego, który dokonuje oceny pozycji.
\subsection{Stockfish}
Stockfish jest najpopularniejszym obecnie używanym silnikiem szachowym, zaprojektowanym przez Marco Costalba, Joona Kiiski, Gary Linscott, Tord Romstad, Stéphane Nicolet, Stefan Geschwentner, and Joost VandeVondele i stale ulepszany jako oprogramowanie typu open-source. Strona \textbf{Lichess.com}\cite{stockfish_lichess} wykorzystuje go do oceny aktualnej pozycji.

Stockfish analizuje poprzez przeszukiwanie za pomocą algorytmu alfa-beta legalne (czyli następujące po ruchu zgodnym z zasadami gry) pozycje, które mogą wyniknąć z aktualnej sytuacji na szachownicy. Dobierają na podstawie najlepszego możliwego zestawu ruchów (zakłada się, że każdy z graczy wykona najlepszy w ocenie silnika ruch) pozycje, które wystąpią dla określonej głębokości (głębokość 18 oznacza 18 ruchów białych i 18 czarnych) i na ich podstawie ocenia aktualną pozycję.

\subsection{Ewaluacja}
Wspomniana wcześniej ewaluacja, wyliczana przez silnik szachowy jest funkcją
\begin{equation}
	f()
\end{equation}
\chapter{ZAGADNIENIE TEORETYCZNE II - użyte metody, teoria stojąca za rozwiązaniami problemów}
\chapter{sformuowanie problemów analitycznych, które chce zbadać}
\chapter{analiza / rozwiązanie problemów}
\section{Dane (co zawierają surowe dane)}
Dane, \textbf{[...]} zostały pobrane z platformy Lichess \cite{lichess}. Są one przechowywane w plikach o rozmiarze kilkudziesięciu Gb. Każdy z nich zawiera wszystkie gry rozegrane na platformie w ciągu całego miesiąca. Ponadto, ok. 7\% gier zostało wcześniej przeanalizowane przez silnik szachowy Stockfish \textbf{WYJAŚNIĆ CZYM JEST STOCKFISH I EVAL ALE TO WCZEŚNIEJ} i posiadają dane punktowe o nazwie \textit{Eval}, określające unormowaną przewagę jednego z graczy. Przykładowy zapis jednej takiej gry został zaprezentowany na rysunku \textbf{XXX}:
\textbf{WSTAWIĆ RYSUNEK Z DANYMI Z EVAL}
\subsection{Odfiltrowanie danych (jak zostają pozyskane)}
\textbf{TUTAJ INFORMATYCZNA CZĘŚĆ O TYM JAK POZYSKAŁEM DANE Z PLIKU}
Pierwszy
\section{Analiza pierwszego problemu}
\section{analiza drugiego problemu...  and so on...}
\chapter{wnioski, podsumowanie}

\chapter{tabelka}
Tabela \ref{tab:przykladowa} 
\begin{table}[H]
	\caption{Podstawowa Tabela}
	\centering
	\begin{tabular}{ccc}
		\hline
		\hline                       
		Państwo & PKB (w milionach USD )& Stopa bezrobocia  \\  [0.5ex] 
		\hline 
		Stany Zjednoczone & 75 278 049 & 4,60\%  \\
		Chiny & 11 218 281 & 4,10\%   \\
		Japonia & 4 938 644 & 3,10\%  \\
		Niemcy & 3 466 639 & 6,00\%   \\
		Wielka Brytania & 2 629 188 & 4,60\%  \\ [1ex]  
		\hline 
	\end{tabular}
	\caption*{\textit{Źródło: opracowanie własne}}
	\label{tab:przykladowa2} 
\end{table}
\chapter{rysunek}
Rysunki do pracy dyplomowej należy wstawiać w sposób podobny do wstawiania tabel, z~zasadniczą różnicą polegającą na tym, że podpis powinno umieszczać się centralnie pod rysunkiem, a nie powyżej niego. Numeracja i sposób cytowania pozostają bez zmian, przy czym tabele i rysunki nie mają numeracji wspólnej, np. po Tabeli \ref{tab:przykladowa2} występuje Rysunek \ref{rys1} (o ile jest to pierwszy rysunek rozdziału pierwszego), a nie Rysunek $1.3$.

\begin{figure}[ht]
	
	\centering
	
	\includegraphics[scale=0.27]{logo_w13.jpg}
	\caption{Podstawowy Rysunek}\label{rys1}
\end{figure}
\label{rys:przykladowy} 


\chapter{Definicje, lematy, twierdzenia, przykłady i wnioski}
Definicje, lematy, twierdzenia, przykłady i wnioski piszemy w pracy tak:
\begin{definition}[Martyngał]
	Tu piszemy treść definicji martyngału.
\end{definition}
\begin{lemma}[]% w nawiasie kwadratowym można napisać jego nazwę
	Tu piszemy treść lematu.
\end{lemma}
\chapter{cytowanie}
Do cytowania używamy komendy \texttt{cite}. W nawiasie klamrowym podajemy klucz, którego użyliśmy w pliku \emph{bibliografia.bib}. Przykład: \cite{einstein} lub \cite[chap. 2]{latexcompanion}.

%{\backmatter \chapter{Podsumowanie}}
%Podsumowanie w pracach matematycznych nie jest obligatoryjne. Warto jednak na zakończenie krótko napisać, co udało nam się %zrobić w pracy, a czasem także o tym, czego nie udało się zrobić.

{\backmatter \chapter{Dodatek}}
Dodatek w pracach matematycznych również nie jest wymagany. Można w nim przedstawić np. jakiś dłuższy dowód, który z pewnych przyczyn pominęliśmy we właściwej części pracy lub (np. w przypadku prac statystycznych) umieścić dane, które analizowaliśmy.

%%%%%%%%%%%%%%%%%%%%%%%%%%%%%%%%%%%%%%%%%%%%%%%%%%%%%%%%%
% BIBLIOGRAFIA
% W tworzeniu bibliografii najlepiej korzystać z BibTex'a, 
% który jest częścią systemu Tex. W naszym przypadku funkcję 
% przechowalni literatury, do której się odwołujemy, pełni 
% plik bibliografia.bib. Nie musimy ręcznie dodawać nowych 
% pozycji do bibliografii. Możemy wejść np. na stronę 
% https://mathscinet.ams.org/mathscinet/index.html, 
% znaleźć odpowiednią pozycję, wybrać ją, a następnie zmienić 
% 'Select alternative format' na BibTeX, skopiować uzyskany 
% tekst, wkleić do pliku bibliografia.bib i skompilować. 
% Gotowe informacje do pliku bibliografia.bib można znaleźć 
% także na https://arxiv.org - gdy znajdziemy interesującą nas 
% pracę, szukamy 'References & Citations' i klikamy 'NASA ADS', 
% a potem 'Bibtex entry for this abstract' 
% i postępujemy tak jak wcześniej.
%%%%%%%%%%%%%%%%%%%%%%%%%%%%%%%%%%%%%%%%%%%%%%%%%%%%%%%%%
\newpage
% w nawiasie klamrowym wpisujemy nazwę pliku z bibliografią w formacie .bib
\bibliography{bibliografia} 
\end{document}